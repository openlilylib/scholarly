\PassOptionsToPackage{unicode=true}{hyperref} % options for packages loaded elsewhere
\PassOptionsToPackage{hyphens}{url}
%
\documentclass[]{ollmanual}
\usepackage{lmodern}
\usepackage{amssymb,amsmath}
\usepackage{ifxetex,ifluatex}
\usepackage{fixltx2e} % provides \textsubscript
\ifnum 0\ifxetex 1\fi\ifluatex 1\fi=0 % if pdftex
  \usepackage[T1]{fontenc}
  \usepackage[utf8]{inputenc}
  \usepackage{textcomp} % provides euro and other symbols
\else % if luatex or xelatex
  \usepackage{unicode-math}
  \defaultfontfeatures{Ligatures=TeX,Scale=MatchLowercase}
\fi
% use upquote if available, for straight quotes in verbatim environments
\IfFileExists{upquote.sty}{\usepackage{upquote}}{}
% use microtype if available
\IfFileExists{microtype.sty}{%
\usepackage[]{microtype}
\UseMicrotypeSet[protrusion]{basicmath} % disable protrusion for tt fonts
}{}
\IfFileExists{parskip.sty}{%
\usepackage{parskip}
}{% else
\setlength{\parindent}{0pt}
\setlength{\parskip}{6pt plus 2pt minus 1pt}
}
\usepackage{hyperref}
\hypersetup{
            pdftitle={The openLilyLib Package scholarLY},
            pdfauthor={Urs Liska},
            pdfborder={0 0 0},
            breaklinks=true}
\urlstyle{same}  % don't use monospace font for urls
\usepackage{color}
\usepackage{fancyvrb}
\newcommand{\VerbBar}{|}
\newcommand{\VERB}{\Verb[commandchars=\\\{\}]}
\DefineVerbatimEnvironment{Highlighting}{Verbatim}{commandchars=\\\{\}}
% Add ',fontsize=\small' for more characters per line
\newenvironment{Shaded}{}{}
\newcommand{\AlertTok}[1]{\textcolor[rgb]{1.00,0.00,0.00}{\textbf{#1}}}
\newcommand{\AnnotationTok}[1]{\textcolor[rgb]{0.38,0.63,0.69}{\textbf{\textit{#1}}}}
\newcommand{\AttributeTok}[1]{\textcolor[rgb]{0.49,0.56,0.16}{#1}}
\newcommand{\BaseNTok}[1]{\textcolor[rgb]{0.25,0.63,0.44}{#1}}
\newcommand{\BuiltInTok}[1]{#1}
\newcommand{\CharTok}[1]{\textcolor[rgb]{0.25,0.44,0.63}{#1}}
\newcommand{\CommentTok}[1]{\textcolor[rgb]{0.38,0.63,0.69}{\textit{#1}}}
\newcommand{\CommentVarTok}[1]{\textcolor[rgb]{0.38,0.63,0.69}{\textbf{\textit{#1}}}}
\newcommand{\ConstantTok}[1]{\textcolor[rgb]{0.53,0.00,0.00}{#1}}
\newcommand{\ControlFlowTok}[1]{\textcolor[rgb]{0.00,0.44,0.13}{\textbf{#1}}}
\newcommand{\DataTypeTok}[1]{\textcolor[rgb]{0.56,0.13,0.00}{#1}}
\newcommand{\DecValTok}[1]{\textcolor[rgb]{0.25,0.63,0.44}{#1}}
\newcommand{\DocumentationTok}[1]{\textcolor[rgb]{0.73,0.13,0.13}{\textit{#1}}}
\newcommand{\ErrorTok}[1]{\textcolor[rgb]{1.00,0.00,0.00}{\textbf{#1}}}
\newcommand{\ExtensionTok}[1]{#1}
\newcommand{\FloatTok}[1]{\textcolor[rgb]{0.25,0.63,0.44}{#1}}
\newcommand{\FunctionTok}[1]{\textcolor[rgb]{0.02,0.16,0.49}{#1}}
\newcommand{\ImportTok}[1]{#1}
\newcommand{\InformationTok}[1]{\textcolor[rgb]{0.38,0.63,0.69}{\textbf{\textit{#1}}}}
\newcommand{\KeywordTok}[1]{\textcolor[rgb]{0.00,0.44,0.13}{\textbf{#1}}}
\newcommand{\NormalTok}[1]{#1}
\newcommand{\OperatorTok}[1]{\textcolor[rgb]{0.40,0.40,0.40}{#1}}
\newcommand{\OtherTok}[1]{\textcolor[rgb]{0.00,0.44,0.13}{#1}}
\newcommand{\PreprocessorTok}[1]{\textcolor[rgb]{0.74,0.48,0.00}{#1}}
\newcommand{\RegionMarkerTok}[1]{#1}
\newcommand{\SpecialCharTok}[1]{\textcolor[rgb]{0.25,0.44,0.63}{#1}}
\newcommand{\SpecialStringTok}[1]{\textcolor[rgb]{0.73,0.40,0.53}{#1}}
\newcommand{\StringTok}[1]{\textcolor[rgb]{0.25,0.44,0.63}{#1}}
\newcommand{\VariableTok}[1]{\textcolor[rgb]{0.10,0.09,0.49}{#1}}
\newcommand{\VerbatimStringTok}[1]{\textcolor[rgb]{0.25,0.44,0.63}{#1}}
\newcommand{\WarningTok}[1]{\textcolor[rgb]{0.38,0.63,0.69}{\textbf{\textit{#1}}}}
\setlength{\emergencystretch}{3em}  % prevent overfull lines
\providecommand{\tightlist}{%
  \setlength{\itemsep}{0pt}\setlength{\parskip}{0pt}}
\setcounter{secnumdepth}{0}
% Redefines (sub)paragraphs to behave more like sections
\ifx\paragraph\undefined\else
\let\oldparagraph\paragraph
\renewcommand{\paragraph}[1]{\oldparagraph{#1}\mbox{}}
\fi
\ifx\subparagraph\undefined\else
\let\oldsubparagraph\subparagraph
\renewcommand{\subparagraph}[1]{\oldsubparagraph{#1}\mbox{}}
\fi

% set default figure placement to htbp
\makeatletter
\def\fps@figure{htbp}
\makeatother


\title{The openLilyLib Package \texttt{scholarLY}}
\author{Urs Liska}
\date{\today}

\begin{document}
\maketitle

{
\setcounter{tocdepth}{3}
\tableofcontents
}
\pagebreak

\hypertarget{introduction}{%
\section{Introduction}\label{introduction}}

\ollPackage{openLilyLib}\footnote{\url{https://github.com/openlilylib}}
(or ``open LilyPond Library'') is an extension system for the GNU
LilyPond\footnote{\url{http://lilypond.org}} score writer. It provides a
plugin infrastructure, a general-purpose toolkit of building blocks, and
a growing number of packages for specific purposes. The main intention
is to encapsulate potentially complex programming and make it available
with a consistent and easy-to-use interface. \textbf{TODO:} Provide a
central source of documentation.

\ollPackage{scholarLY}\footnote{\url{https://github.com/openlilylib/scholarly}}
is a package dedicated to the needs of scholarly editors, although its
tools have proven useful in general-purpose applications too, as a kind
of ``in-score communication system'' or ``in-score issue tracker''. The
main concerns of scholarly editions and workflows addressed by the
package are:

\begin{itemize}
\tightlist
\item
  \emph{Handling annotations}\\
  The package makes it possible to maintain the whole critical apparatus
  within the source files of the score itself, providing point-and-click
  navigation between score and annotations. These annotations can be the
  immediate source of professionally typeset critical reports.\footnote{Planned
    is also the support for attaching annotations directly to the
    graphical objects in SVG output and providing the means for
    interactive display of annotations.}\\
  This is handled by the \ollPackage{scholarly.annotate} module.
\item
  \emph{Encoding source findings and editorial decisions}\\
  The package provides commands to tag music with semantic editorial
  markup as part of the scholarly editor's duty to document their
  findings and decisions. It is possible to simply \emph{document} these
  cases and get visual feedback through colors during the editing
  process, but it is also possible to persistently apply styles to
  arbitrary cases (for example: parenthesize editorial additions).\\
  This is handled by the \ollPackage{scholarly.editorial-markup} module.
\item
  \emph{Encode alternative texts}\\
  Typically editorial processes involve decisions to either choose from
  various readings or to make emendations to a text found in the
  sources. It is part of the duty to a scholarly editor to document all
  these cases, which traditionally is done in textual form in the
  critical report. Digital editing techniques enjoy the possibility to
  directly \emph{encode} such differences, and the package provides the
  necessary tools to do so. The encoding of alternative texts can be
  simply used as a means of \emph{documentation}, but it can also
  produce alternative \emph{renderings} of an edition.\\
  This is handled by the \ollPackage{scholarly.choice} module.
\end{itemize}

\hypertarget{installation-and-dependencies}{%
\subsection{Installation and
Dependencies}\label{installation-and-dependencies}}

The installation of openLilyLib and its packages is described in the
\ollPackage{oll-core} documentation.\footnote{\url{https://github.com/openlilylib/oll-core/wiki}}
The code for the \ollPackage{scholarLY} package may be cloned or
downloaded from \texttt{https://github.com/openlilylib/scholarly}.

\ollPackage{scholarLY} depends on the following openLilyLib packages
that have also to be installed:

\begin{itemize}
\tightlist
\item
  oll-core\footnote{\url{https://github.com/openlilylib/oll-core}}
\item
  stylesheets\footnote{\url{https://github.com/openlilylib/stylesheets}}
\end{itemize}

To make \ollPackage{scholarLY} available to a LilyPond document first
include \ollPackage{oll-core}, then load the package with
\cmd{loadPackage} or an individual module with \cmd{loadModule}

\begin{Shaded}
\begin{Highlighting}[]
\KeywordTok{\textbackslash{}include}\NormalTok{ "}\StringTok{oll-core/package.ily"}
\FunctionTok{\textbackslash{}loadPackage}\NormalTok{ \textbackslash{}with}\KeywordTok{ \{}
\NormalTok{  modules = annotate.editorial-markup.choice}
\KeywordTok{\}}
\end{Highlighting}
\end{Shaded}

\hypertarget{the-annotate-module}{%
\section{\texorpdfstring{The \texttt{annotate}
Module}{The annotate Module}}\label{the-annotate-module}}

\ollPackage{scholarly.annotate} is the core of the
\ollPackage{scholarLY} package, providing its most prominent feature
with the handling of annotations and critical apparatus. It came into
existence with the goal of overcoming or at least alleviating annoying
limitations of traditional toolchains and workflows.

One of the most ``sacred'' duties of scholarly editors is not to
determine the perfect text but to transparently document and explain the
rationale behind the edited text, describing variant readings in the
source(s) and revealing the observations the decisions are based upon.
This is traditionally done in textual form -- occasional music examples
notwithstanding -- in critical reports that live in separate documents
from the score.

In a typical setting an editor is facing three separate entities: one or
multiple sources, the score being created, and the critical
observations. There are various ways to organize this, but the awkward
reality is that the entities are materially separate and not linked.
Keeping them synchronized during the editing process is a tedious and
error-prone effort. A typical situation while proof-reading is the
evaluation of an observed difference between the source and the new
score: first the editor has to look up the corresponding measure in
their critical remarks and determine if this difference has been
documented already. If this is the case they can continue -- but more
often than not they will have to repeat this lookup in any subsequent
run-through since there usually is no visual indication in the new
score. If there is \emph{no} annotation the editor has to decide whether
they have to add an annotation, keep it as an undocumented change, or
change the text of the new score (optionally adding an annotation
anyway).

As critical editions may involve hundreds or thousands of such instances
proof-reading can amount to an unnerving sequence of context switches,
with each switch being complicated by the lack of synchronization
between the different entities. \ollPackage{scholarly.annotate}
significantly reduces the complexity of the task by having the critical
annotations encoded directly in the score files while providing visual
feedback and point-and-click navigation. Additionally critical reports
can be generated and professionally typeset directly from these encoded
annotations, avoiding the effort of keeping the report up to date and
the measure numbers in sync.

Annotations are encoded within the LilyPond input files, right next to
the score element they refer to. This means the documentation of the
editorial process is maintained together with the edition data itself.
Through different annotation types it is possible to discern between
various stages of the process and definitive critical remarks that are
intended to be printed in the reports.

Two different feedback channels provide convenient access to
annotations. Annotations are printed in the console/log output, giving a
convenient list to review all annotations. In editing environments like
\emph{Frescobaldi}\footnote{\url{http://frescobaldi.org}} this can be
done with a simple key combination, immediately positioning the input
cursor at the annotations. By default annotations also highlight the
annotated element through colors, making them immediately obvious when
browsing the score, for example while proof-reading. Clicking on the
annotated elements again places the input cursor at the annotation.
There are plans to add GUI support for annotation browsing and editing
in Frescobaldi.

Finally annotations can be exported to various file formats, creating
the basis for external tools to create reports from. There are output
routines for plain text and \LaTeX~so far, and HTML export is in
development. Other formats are planned and can easily be plugged into
the infrastructure.

\begin{itemize}
\tightlist
\item
  \textbf{TODO:} Create and insert screenshot
\end{itemize}

\hypertarget{creating-annotations}{%
\subsection{Creating Annotations}\label{creating-annotations}}

Annotations can be created in two different ways, first by issuing one
of the explicit annotation commands, second by turning a \cmd{tagSpan}
or \cmd{editorialMarkup} into an annotation. Technically these are
equivalent since annotations directly build upon the functionality of
\cmd{tagSpan}, but there is a conceptual difference that should be
considered on a case-by-case basis. An annotation ``annotates'' a score
element or some music while spans or editorial markup ``tag'' that music
``as something'' (for example an editorial addition) and can be
annotated \emph{on top} of that. The current chapter exclusively uses
explicit commands but keep this fact in mind when reading the chapters
about the \ollPackage{editorial-markup} module.

The syntax for creating annotations is one of these equivalent
invocations:

\begin{Shaded}
\begin{Highlighting}[]
\StringTok{\textbackslash{}<}\NormalTok{annotation-command}\ErrorTok{>}\NormalTok{ \textbackslash{}with}\KeywordTok{ \{}\NormalTok{ <attribute> = <value> ... }\KeywordTok{\}} \DataTypeTok{<}\NormalTok{music>}
\FunctionTok{\textbackslash{}tagSpan}\NormalTok{ annotation \textbackslash{}with}\KeywordTok{ \{}\NormalTok{ <attribute> = <value> ... }\KeywordTok{\}} \DataTypeTok{<}\NormalTok{music>}
\FunctionTok{\textbackslash{}editorialMarkup} \DataTypeTok{<}\NormalTok{markup-type> \textbackslash{}with}\KeywordTok{ \{}\NormalTok{ ann-type = <annotation-type> ... }\KeywordTok{\}} \DataTypeTok{<}\NormalTok{music>}
\FunctionTok{\textbackslash{}tagSpan} \DataTypeTok{<}\NormalTok{arbitrary-name> \textbackslash{}with}\KeywordTok{ \{}\NormalTok{ ann-type = <annotation-type> ... }\KeywordTok{\}} \DataTypeTok{<}\NormalTok{music>}
\end{Highlighting}
\end{Shaded}

All of these can also be applied as post-event functions (prepending the
leading backslash with a directional operator), which is discussed
shortly.

Note that adding the \option{ann-type} attribute to a span (or
editorial-markup, which technically \emph{is} a span) will \emph{create}
an annotation, independently of the \ollPackage{scholarLY} module.
However, only when that module is loaded annotations will also be
\emph{processed}.

There are various ways to apply annotations to music, and there are many
things to know about configuring annotations with attributes (content)
and options (processing), but these are described in later sections.

\hypertarget{the-five-annotation-types}{%
\subsection{The Five Annotation Types}\label{the-five-annotation-types}}

There are five types of annotations\footnote{It is possible to add
  custom annotation types, but this is somewhat involved, and there is
  no convenient interface available for it yet. Essentially the new type
  has to be registered in a number of places and suitable defaults and
  handler functions defined.}, used for different tasks in the editing
process. The following list gives both the command name and the
corresponding \option{ann-type} attribute value:

\begin{itemize}
\tightlist
\item
  \cmd{criticalRemark} (\option{critical-remark})\\
  Documents a definitive editorial finding or decision.
\item
  \cmd{musicalIssue} (\option{musical-issue})\\
  Points to an editorial observation that is considered an open question
  and yet has to be finalized
\item
  \cmd{lilypondIssue} (\option{lilypond-issue})\\
  Highlight technical issues that need to be resolved
\item
  \cmd{question} (\option{question})\\
  \cmd{todo} (\option{todo})\\
  General-purpose annotations
\end{itemize}

Critical remarks and musical issues are typically used as inherent parts
of a scholarly workflow. It is recommended practice to generously add
\cmd{musicalIssue} annotations for any observations and distill them to
a more concise set of \cmd{criticalRemark} entries throughout the
process. This concept also has proven very efficient when applied to
workflows with peer review.

As these scholarly annotations usually refer to source findings and
decisions it is appropriate to use them as part of \cmd{editorialMarkup}
entries, while the other three annotation types lend themselves more to
general-purpose ``in-source communication'' or ``issue tracker'' usage
and are therefore more inclined to be used as standalone annotations
(note that standalone annotations also can annotate \emph{spans} of
music).

\ollIssue{Deprecation!}

With \ollPackage{scholarLY} version 0.6.0 the implementation of
annotations has been fundamentally rewritten. This led to a breaking
change in syntax while the command names have been kept.

The old implementation of these commands is still available but has been
moved to the \ollPackage{scholarly.annotate.legacy} module. If you have
used \ollPackage{scholarLY} with the old interface and don't want to
immediately update your code you have to change the \cmd{loadModule}
invocation to \cmd{loadModule scholarly.annotate.legacy} -- which of
course will prevent you from making use of any new functionality.

When the \texttt{item} after the \cmd{with} block is \texttt{NoteHead}
it is usually sufficient to simply remove that, but in other cases it
may be more complex, even if the result is more concise.

Apart from the incompatible input syntax the most significant difference
is the way how visual styling functions are applied to the annotated
music. In the \ollPackage{legacy} module an annotation can/could be told
to \texttt{apply} an editorial function -- if one is defined. In the new
implementation the task of applying styling functions is built into the
spans themselves (both \cmd{tagSpan} and \cmd{editorialMarkup}). So the
``editorial-command'' is now managed by the ``span'', and the annotation
is created on top of that already-styled span, rather than having an
annotation ask for the application of a styling function.

\hypertarget{application-of-annotations}{%
\subsection{Application of
Annotations}\label{application-of-annotations}}

Annotations can be applied to sequential or single music expressions or
as post-events. As has been said annotations build upon the
\cmd{tagSpan} command and share their behaviour with regard to their
application to some music. Therefore more details on that topic can be
obtained from the \ollPackage{stylesheets} manual. While annotations can
affect sequential music expressions (which is visible by the coloring)
they are technically attached to the first element in them. This single
or first element determines the reported musical moment of the
annotation.

The following examples should give a sufficient overview of the options
and possibilities. They are limited to the most basic content, namely
the mandatory \option{message} attribute and in one case the
\option{item} attribute to affect a dedicated score element type. A last
example is used to demonstrate the creation of footnotes and is included
here because footnotes can only be displayed in a fullpage example.

\begin{Shaded}
\begin{Highlighting}[]

\end{Highlighting}
\end{Shaded}

\lilypondfile[insert=fullpage,nofragment,indent=0cm,debug]{annotations-page.ly}

\hypertarget{authoring-annotation-attributescontent}{%
\subsection{Authoring Annotation
(Attributes/Content)}\label{authoring-annotation-attributescontent}}

The \emph{content} of annotations is defined through the attributes in a
\cmd{with
\{\}} block, regardless of the way the annotation is applied to the
music. There are \emph{mandatory} attributes, \emph{known} attributes,
\emph{auto-generated} attributes, and the option to use arbitrary
\emph{custom} attributes. We'll go through each of these groups in turn.

\hypertarget{mandatory-attributes}{%
\subsubsection{Mandatory Attributes}\label{mandatory-attributes}}

\ollOption{ann-type}{}
\ollOption{message}{}

The presence of two attributes is fundamental to making an annotation:
\option{ann-type} and \option{message}. The \option{ann-type} attribute
is what actually makes an annotation an annotation. Every ``span'' has
an annotation attached, but only annotations with a dedicated type will
be processed by the annotation engraver. The \option{ann-type} must be
one of the five recognized annotation types, and the explicit annotation
commands transparently set this attribute.

When an annotation is created the presence of \option{message} is
mandatory. It is a free-form text that is printed to the console and
exported to files. In addition it is used as a fallback value for some
other attributes. If it is not explicitly given a default value is
supplied.

\hypertarget{known-attributes}{%
\subsubsection{Known Attributes}\label{known-attributes}}

Annotations themselves don't check for selection and type of attributes,
basically \emph{anything} can be supplied by users. However, there is a
number of ``known attributes'', i.e.~attributes that have special
meaning for spans or annotations. Generally all attributes handled by
\ollPackage{stylesheets.span} are available in annotations too, namely
options to trigger footnotes, music examples, or balloon text
annotations. Details can be found in the manual for
\ollPackage{stylesheets}.

Secondary code such as custom or public libraries may decide to
recognize and handle additional attributes, making them ``known
attributes'' in \emph{their} context. For example,
\ollPackage{scholarly.editorial-markup} provides a rule-set for handling
additional attributes. \ollPackage{scholarly.annotate} also has its own
known attributes which are used in annotation export and therefore
documented in a later chapter.

\hypertarget{custom-attributes}{%
\subsubsection{Custom Attributes}\label{custom-attributes}}

It is possible to add arbitrary attributes to an annotation, as long as
the name doesn't conflict with other known attributes. By itself the
annotation does not process such attributes but passes it through to the
exported annotations.

Custom attributes may be used in two stages: they can be evaluated by
styling functions (working on the level of \cmd{tagSpan}), or they can
be used from the exported files, for example by a \LaTeX~package
typsetting critical reports. The \option{lycritrprt}\footnote{\url{https://github.com/uliska/lycritrprt}}
package aims at providing a convenient interface to processing such
attributes through a templating system.

\hypertarget{generated-attributes}{%
\subsubsection{Generated Attributes}\label{generated-attributes}}

When an annotation is processed by \ollPackage{annotate}'s engraver some
attributes are added and others are enriched with data that is only
available in that stage of the LilyPond compilation process. The
annotation engraver uses context information to provide additional data
to be stored in the annotation and eventually exported, so it is
important to know about the underlying mechanisms in order to properly
set up the scores.

\ollOption{context-label}{}

The \emph{context id} is used to specify the ``context'' -- usually the
instrument/voice -- an annotation refers to. Initially this attribute is
set to the value
\texttt{\textless{}directory\textgreater{}.\textless{}file\textgreater{}},
so it is at least known in which \emph{file} an annotation has been
defined. However, the \emph{engraver} may narrow this down to a more
specific and especially \emph{musical} identification. If the annotated
music lives within a named \texttt{Staff} context (i.e.~within an
explicit \cmd{new} \cmd{Staff = "<some-voice-name>"}) the
\option{context-id} attribute is assigned this name
``\textless{}some-voice-name\textgreater{}''.

This information can be used in critical reports printed from the
exported annotations. However, if the LilyPond staff names are not
suitable as human-readable labels in a report it is possible to map
staff names to printable names using

\begin{Shaded}
\begin{Highlighting}[]
\FunctionTok{\textbackslash{}annotateSetContextName} \DataTypeTok{<}\NormalTok{context-id> }\DataTypeTok{<}\NormalTok{label>}
\FunctionTok{\textbackslash{}annotateSetContextName}\NormalTok{ staff-vln-III "}\StringTok{Vl. 3"}
\end{Highlighting}
\end{Shaded}

The resulting string or the original attribute is stored in the
\option{context-label} attribute for displaying/exporting purposes.

\ollOption{score-label}{}

In a similar manner the \option{score-id} attribute reports a score's
name if it has explicitly been named through
\cmd{new Score = "my-score-name"}. If \emph{no} explicit score-id has
been set this property defaults to \texttt{\#f}. This information
becomes relevant when an input file contains multiple scores,
e.g.~movements. A report may use this attribute to filter or group
annotations.

Also like with context-id it is possible to register display names for
scores through

\begin{Shaded}
\begin{Highlighting}[]
\FunctionTok{\textbackslash{}annotateSetScoreName} \DataTypeTok{<}\ErrorTok{s}\NormalTok{core-id> }\DataTypeTok{<}\NormalTok{label>}
\FunctionTok{\textbackslash{}annotateSetScoreName}\NormalTok{ 03-adagio "}\StringTok{Third movement - adagio"}
\end{Highlighting}
\end{Shaded}

The results are stored in the \option{score-label} property. It is
important to realize that the separation of id and label makes it
possible to group annotations by score/movement and have the movements
sorted by a different key (the \texttt{score-id}) than they are
displayed.

\ollOption{grob-label}{}

As described earlier it is possible to target specific grob types
through an annotation's \option{item} attribute. However, whether
explicitly or implicitly, eventually an annotation is always attached to
a \emph{specific} score element (a ``grob''), and its type is made
available as the \option{grob-type} attribute.

This attribute is \emph{always} set, other than \option{item} which only
holds a value when set \emph{explicitly}. Therefore it may be a useful
property for display in reports. Note that in sequential music
expressions the \texttt{grob-type} is the type of the ``anchor'',
i.e.~the first music event in the expression while in chords it is the
first \emph{note} within the chord.

\option{grob-type} always carries the name as used by LilyPond, but it
is possible to add a mapping to speaking (or translated) names through

\begin{Shaded}
\begin{Highlighting}[]
\FunctionTok{\textbackslash{}annotateSetGrobName} \DataTypeTok{<}\NormalTok{grob-type> }\DataTypeTok{<}\NormalTok{label>}
\FunctionTok{\textbackslash{}annotateSetGrobName}\NormalTok{ NoteHead "}\StringTok{Notenkopf"}
\end{Highlighting}
\end{Shaded}

For this attribute there is also the command \cmd{annotateSetGrobNames}
to provide labels for multiple grob types at once:

\begin{Shaded}
\begin{Highlighting}[]
\FunctionTok{\textbackslash{}annotateSetGrobNames}
\NormalTok{#}\FloatTok{'((NoteHead . "}\StringTok{Notenkopf"}\FloatTok{)}
\FloatTok{   (Hairpin . "}\StringTok{Gabel"}\FloatTok{)}
\FloatTok{   (Slur . "}\StringTok{Bogen"}\FloatTok{))}
\end{Highlighting}
\end{Shaded}

\ollOption{grob-location}{}

The \option{grob-location} attribute holds detailed information about
the annotation's moment in musical time. It is an association list with
the following keys:

\begin{itemize}
\tightlist
\item
  \texttt{beat-string}
\item
  \texttt{beat-fraction}
\item
  \texttt{beat-part}
\item
  \texttt{our-beat}
\item
  \texttt{measure-pos}
\item
  \texttt{measure-no}
\item
  \texttt{rhythmic-location}
\item
  \texttt{meter}
\end{itemize}

These fields can be retrieved by secondary code but are mostly used
internally during the export stage. This is described later in this
manual.

\hypertarget{processing-annotations}{%
\subsection{Processing Annotations}\label{processing-annotations}}

So far we have discussed how annotations are created and filled with
content. But of course the second part of the equation is equally
important: processing and export of annotations.

After the annotations have been recorded they are used for the output
stage, which includes the following steps (all can be switched on and
off by configuration options):

\begin{itemize}
\tightlist
\item
  Highlighting the annotated element through colors
\item
  Printing to the console
\item
  Exporting to various file formats
\end{itemize}

The first two are useful for reviewing and navigating the score while
editing, and the third is used to produce definitive reports that are
automatically in sync with the actual score.

\hypertarget{some-technical-background-and-installation-considerations}{%
\subsubsection{Some Technical Background and ``Installation''
Considerations}\label{some-technical-background-and-installation-considerations}}

When the \ollPackage{scholarly.annotate} module is loaded two
\emph{engravers} are implicitly loaded too: \texttt{annotationCollector}
and \texttt{annotationProcessor}. The first is responsible to inspect
all score objects that have an input annotation attached and create an
annotation list from them, the second processes that list, prints the
annotations to the console and exports them to file(s).

\textbf{TODO:} The following has to be reviewed: I think the collector
should live on the voice level. IIRC it was implemented the way it is
because the named contexts are usually on Staff level, but it should of
course be possible to find a better solution by recursively checking
parent contexts (similar to \texttt{score-id}).

\begin{quote}
The collector essentially works on the staff (not voice) level and is
``consisted'' to
\texttt{Staff,\ DrumStaff,\ RhythmicStaff,\ TabStaff,\ GregorianTranscriptionStaff,\ MensuralStaff,\ VaticanaStaff,\ Dynamics,\ Lyrics},
and \texttt{FiguredBass}. The processor is consisted in the
\texttt{Score} context. The latter is not configurable but it is
possible to \cmd{consist} or \option{\textbackslash remove}
\texttt{annotationCollector} from arbitrary contexts on a score-by-score
basis.
\end{quote}

\textbf{TODO:} When the previous has been decided and implemented also
document the option to manually change the context assignments.

\hypertarget{sorting-and-filtering-output-export-targets}{%
\subsubsection{Sorting and Filtering Output, Export
Targets}\label{sorting-and-filtering-output-export-targets}}

The most fundamental configuration of the annotation handling is the
decision what is going to be exported and to which target(s). This is
done through configuration options with
\cmd{setOption scholarly.annotate.<option> <value>} or
\cmd{setChildOption scholarly.annotate.<main-option> <sub-option> <value>}.

\hypertarget{export-targets}{%
\paragraph{Export targets}\label{export-targets}}

\ollOption{scholarly.annotate.print}{\#\#t}

If this option is turned on then annotations are printed to the console.
There they will display either a selection of or all attributes of the
annotation. They will be printed as a ``warning'', which produces a
clickable link in Frescobaldi's log window where annotations can also be
browsed by the \texttt{Ctrl+E} keyboard shortcut.

The option is active by default, reasons to deactivate it would include
use on a server or -- more realistically -- avoiding a cluttered log
output.

\ollOption{scholarly.annotate.export-targets}{\#'()}

A list with targets for which annotations will be produced and exported.
Supported export targets are so far \texttt{plaintext} and
\texttt{latex}, but we hope to add others to this list, such as
\texttt{html} (in progress), \texttt{markdown} or maybe even
\texttt{pdf} (through Pandoc/\LaTeX?).

\begin{Shaded}
\begin{Highlighting}[]
\FunctionTok{\textbackslash{}setOption}\NormalTok{ scholarly.annotate.export-targets plaintext.latex}
\end{Highlighting}
\end{Shaded}

\hypertarget{organizing-annotations-for-output}{%
\paragraph{Organizing Annotations for
Output}\label{organizing-annotations-for-output}}

\ollOption{scholarly.annotate.ignored-types}{\#'()}

A list of annotation types (\texttt{critical-remark} etc.) that should
be ignored for processing. Annotations of ignored type are skipped in an
early stage of the processing, so in large projects it may be efficient
to ignore all types that are not needed.

By default no types are ignored, i.e.~all types are processed.

\begin{Shaded}
\begin{Highlighting}[]
\FunctionTok{\textbackslash{}setOption}\NormalTok{ scholarly.annotate.ignored-types question.todo.lilypond-issue}
\end{Highlighting}
\end{Shaded}

\ollOption{scholarly.annotate.sort-by}{\#'(rhythmic-location)}

Annotations are exported in sorted order, by default according to
musical time. With this option one or multiple sort criteria can be
specified, currently supported orders are:

\begin{itemize}
\tightlist
\item
  \texttt{rhythmic-location} (default) -- sort by musical time
\item
  \texttt{type} -- sort by annotation type
\item
  \texttt{author} -- sort by author (This will fail if any annotation
  does \emph{not} have an \texttt{author} attribute)
\item
  \texttt{score} -- sort by score.id\\
  This is only relevant if more than one score is present in the current
  document and if all scores are explicitly named (otherwise compilation
  will fail). In order to get meaningful results it is recommended to
  name the scores accordingly (e.g. \texttt{01-presto},
  \texttt{02-adagio}) and use \cmd{annotateSetScoreName} to map this to
  a ``speaking'' label.
\item
  \texttt{context} -- sort by context-id\\
  In order to get meaningful results it is also recommended to name each
  context in a sortable fashion and use \cmd{annotateSetContextName} to
  get ``speaking'' labels.
\end{itemize}

Note that annotations may be sorted by \ollPackage{scholarly.annotate}
or at a later stage by a ``consumer''. It may depend on the context or
necessity which approach provides more functionality or is more
efficient, but in general it should be avoided to sort annotations
\emph{both} in LilyPond and later. This means that if you intend to sort
annotations in a later stage it may be useful to explicitly avoid
sorting in LilyPond by setting the option to the empty list:

\begin{Shaded}
\begin{Highlighting}[]
\FunctionTok{\textbackslash{}setOption}\NormalTok{ scholarly.annotate.sort-by #}\FloatTok{'()}
\end{Highlighting}
\end{Shaded}

\hypertarget{appearance-of-the-output}{%
\subsubsection{Appearance of the
Output}\label{appearance-of-the-output}}

\ollOption{scholarly.annotate.use-colors}{\#\#t}

If this option is active annotations are indicated through the
annotation type's color. This behaves differently for annotations
created through explicit annotation commands or those created by adding
the \option{ann-type} attribute to a \cmd{tagSpan}.

In ``real'' annotations the whole span is colored in the annotation
type's color while in implicit annotations the span is colored in the
\emph{span's} color and only the ``anchor'' then colored with the
annotation's color. However, if the span includes only a single element
the annotation color completely overwrites the span color.

\begin{Shaded}
\begin{Highlighting}[]

\end{Highlighting}
\end{Shaded}

\%\lilypondfile{annotate-color.ly}

\ollOption{scholarly.annotate.colors}{\dots}
\ollLilyfuncdef{annotateSetColor}{type color}{}

The colors used for annotation types are stored in the
\option{scholarly.annotate.colors} option and can be changed with
\cmd{annotateSetColor <type> <color>}. By default critical remarks are
dark green, musical issues green, LilyPond issue red, questions blue and
todo items magenta. Please note that the use of the default colors makes
them immediately obvious to other users of the package, and you should
only change them for good reasons or if you are sure that you won't
share your documents with others.

\ollOption{scholarly.annotate.attribute-labels}{}
\ollLilyfuncdef{annotateSetAttributeLabel}{name label}{}

When printing annotations to the console or exporting to plaintext
attribute names are replaced with labels if defined in this option, with
default labels provided for all relevant known attributes. Setting
custom labels can be relevant for custom attributes or for translating
the interface.

\ollOption{scholarly.annotate.export.skip-attributes}{}
\ollLilyfuncdef{annotateShowAllAttributes}{on (boolean?)}{\#\#f}

A list with attributes to suppress in output is stored in
\option{scholarly.annotate.export.skip-attributes}. For customization
purposes this the option may be overridden manually, but in general the
interface for this is the function
\cmd{annotateShowAllAttributes <\#\#t/\#\#f} that toggles concise and
verbose output.

\ollOption{scholarly.annotate.context-names}{\#'()}
\ollLilyfuncdef{annotateSetContextName}{context display-name}{}

\ollPackage{scholarly.annotate} uses explicit context names for
displaying when possible. However, it may not always be desirable to use
speaking names for contexts, for example when they have to be generated,
or when annotations should be sorted in score order (e.g.~from top to
bottom). For this case labels can be mapped through the
\option{scholarly.annotate.context-names} option and set through
\cmd{annotateSetContextName <context-name> <context-label>}

\begin{Shaded}
\begin{Highlighting}[]
\FunctionTok{\textbackslash{}annotateSetContextName}\NormalTok{ #}\FloatTok{'}\DecValTok{04}\FloatTok{-flute-1}\NormalTok{ "}\StringTok{Flute 1"}
\end{Highlighting}
\end{Shaded}

\ollOption{scholarly.annotate.score-names}{\#'()}
\ollLilyfuncdef{annotateSetScoreName}{context display-name}{}

With the same rationale as with the context names display names for
scores can be registered with
\cmd{annotateSetScoreName <score-id> <score-label>}

\ollOption{scholarly.annotate.grob-names}{\#'()}
\ollLilyfuncdef{annotateSetGrobName}{grob-type label}{}
\ollLilyfuncdef{annotateSetGrobNames}{mappings}{}

Each annotation carries the name of the affected grob type, in the form
used by LilyPond. This option and two corresponding commands associate
speaking labels with grob types for use in reports.

\begin{Shaded}
\begin{Highlighting}[]
\FunctionTok{\textbackslash{}annotateSetGrobName}\NormalTok{ NoteHead "}\StringTok{Notenkopf"}
\FunctionTok{\textbackslash{}annotateSetGrobNames}
\NormalTok{#}\FloatTok{'((Rest . "}\StringTok{Pause"}\FloatTok{)}
\FloatTok{   (Flag . "}\StringTok{Notenhals"}\FloatTok{)}
\FloatTok{   (Beam . "}\StringTok{Balken"}\FloatTok{))}
\end{Highlighting}
\end{Shaded}

\ollOption{scholarly.annotate.export.type-labels}{}
\ollLilyfuncdef{annotateSetTypeLabel}{type label}{}

Labels for the annotation type. This is not used in \LaTeX~output
because for this target explicit \LaTeX~command names are required.

\hypertarget{exporting-annotations}{%
\subsection{Exporting Annotations}\label{exporting-annotations}}

When exporting annotations they will be stored in a file with the name
\option{<file-basename>.annotations.<extension>}. This is not
configurable so far. Various options that affect the way how annotations
are exported in general have been described above, the following
sections provide details about specific export targets.

\hypertarget{printing-to-the-console}{%
\subsubsection{Printing to the Console}\label{printing-to-the-console}}

All options that are currently supported for printing on the console
have been described above.

\hypertarget{export-to-plaintext}{%
\subsubsection{Export to Plaintext}\label{export-to-plaintext}}

\hypertarget{export-to}{%
\subsubsection{\texorpdfstring{Export to
\LaTeX}{Export to }}\label{export-to}}

\ollOption{scholarly.annotate.export.latex.commands}{}

Command names to be used when exporting to \LaTeX.

\ollOption{scholarly.annotate.export.latex.use-lilyglyphs}{\#\#f}

If this option is set to \#\#t the annotation's musical moment is
exported (to \LaTeX) as a \ollPackage{lilyglyphs}\footnote{\textless{}https://github.com/uliska/lilyglyphs}
command. This \LaTeX package (generally available, e.g.~in \TeX\{\}
Live) provides LilyPond's notational elements to be included in
continuous text, when used with \ollPackage{scholarly.annotate} it uses
musical symbols to denote the display the musical moment of the
annotation.

\hypertarget{the-editorial-markup-module}{%
\section{\texorpdfstring{The \texttt{editorial-markup}
Module}{The editorial-markup Module}}\label{the-editorial-markup-module}}

\ollPackage{scholarly.editorial-markup} provides tools to encode and
visualize source findings and editorial decisions. It is inspired by
corresponding sections of the MEI specification\footnote{\url{http://music-encoding.org/guidelines/v3/content/}}
\footnote{\url{http://music-encoding.org/guidelines/v3/content/critapp.html}}
\footnote{\url{http://music-encoding.org/guidelines/v3/content/edittrans.html}}
but has been adapted to LilyPond's use case. Internally it builds upon
the \ollPackage{stylesheets.span} module and provides specializations
for the specific application.

\ollPackage{scholarly.editorial-markup} provides the
\cmd{editorialMarkup} command to encode and visualize source findings
and editorial decisions. Modeled after MEI, the de-facto standard of
(scholarly) digital music editing, it aims at a unified interface for
sharing scholarly workflows and -- hopefully -- data.

Technically \cmd{editorialMarkup} is a thin wrapper around \cmd{tagSpan}
which is defined in \ollPackage{stylesheets.span}, and it essentially
provides a specialized version of that command, tailoring a deliberately
chosen set of span classes, some additional validation rules, and
default colors. As is documented with the \ollPackage{span} module,
adding an \option{ann-type} attribute triggers the creation of an
annotation, which is typically what one wants when using editorial
markup. The \ollPackage{scholarly.choice} module provides some
additional machinery for that.

Not yet implemented but an integral part of the concept is an
infrastructure for creating and managing styling functions suitable for
the scholarly purpose.

Typically \cmd{editorialMarkup} is used within \cmd{choice} from the
\ollPackage{scholarly.choice} module to encode alternative versions of a
musical text.

\hypertarget{the-command}{%
\subsection{\texorpdfstring{The \cmd{editorialMarkup}
Command}{The  Command}}\label{the-command}}

The syntax and usage of \cmd{editorialMarkup} exactly matches that of
\cmd{tagSpan}:
\texttt{\textbackslash{}editorialMarkup\ \textless{}span-class\textgreater{}\ (\textless{}attributes\textgreater{})\ \textless{}music\textgreater{}},
the only difference being that
\texttt{\textless{}span-class\textgreater{}} may not be an
\emph{arbitrary} name but must be one out of the list of predefined
types as described below. The mechanism of applying the function to some
music is identical, and so is the mechanism to provide custom styling
functions. The only difference is that validator functions have been
provided to match and enforce the predesigned data model of scholarly
editions. While it is \emph{possible} to override the validators with
custom functions it is strongly discouraged.

\begin{Shaded}
\begin{Highlighting}[]

\end{Highlighting}
\end{Shaded}

\lilypondfile{editorial-markup-basic.ly}

\hypertarget{span-classes-defined-by-the-module}{%
\subsection{Span Classes Defined By The
Module}\label{span-classes-defined-by-the-module}}

This section documents the allowed span-classes for
\cmd{editorialMarkup}. They refer to elements defined in the MEI
specification, which is discussed with each class. Some classes have
rules about specific attributes while others are neutral in this
respect. In some cases default attribute names from MEI are suggested
but not enforced.

Note the in the majority of cases \cmd{editorialMarkup} will be used
within \cmd{choice}, and most classes form natural pairs or groups with
other classes. However, all of them may also be used standalone.
Consider the basic case of an apparent error like the a\flat~in the
example above. An edition could silently correct the error, correct the
error but identify the correction, print the original text but mark it
up as erroneous, or it could encode both, giving a choice.

\hypertarget{generic-attributes}{%
\subsubsection{Generic Attributes}\label{generic-attributes}}

MEI defines a number of generic attributes that can be applied to
arbitrary elements. Projects using
\ollPackage{scholarly.editorial-markup} are encouraged to make use of
these attributes and enforce them. At least they should use the
standardized names if they make use of the functionality, rather than
inventing their own.

\ollOption{source}{}

The musical source to which the encoding applies. This may either be a
literal string or a reference to an entry in the sources list
\textbf{TODO:} \emph{This has to be considered and implemented!}
(\texttt{stemma} module?)

\ollOption{certainty}{}

Indicates the level of certainty attributed to the finding. If the
option \option{scholarly.certainty-levels} contains a list with values
only values from this list are allowed. \textbf{TODO:} \emph{This has to
be implemented!}

\ollOption{responsible}{}

Indicates who is responsible for the encoded fact. If the span tags a
source finding the responsibility points to the person that is
presumably responsible for what is found in the source, if an editorial
decision is tagged, the responsibility refers to the \emph{current}
editor.

\ollOption{agent}{}

This too indicates a responsibility, but rather than a person it is
usually meant to refer to tools or other external forces (an
\texttt{agent} could be ``razor'', ``dust'', ``age'' etc.).

\ollOption{type}{}

While some classes \emph{require} a \texttt{type} attribute it may
freely be used with any classes.

\ollOption{reason}{}

A short phrase (shorter than the \texttt{message}) arguing about the
reason of a finding.

\hypertarget{span-classes}{%
\subsubsection{Span Classes}\label{span-classes}}

\ollOption{lemma}{}
\ollOption{reading}{}

Used with \cmd{choice variants}.

Alternative readings from different sources. \texttt{lemma} is the
reading chosen by the editor while \texttt{reading} encodes the reading
from a secondary source. Both classes require the \option{source}
attribute. A \option{sequence} attribute may be used to encode the
(assumed) order in the genesis of the work.

\ollOption{addition}{}
\ollOption{deletion}{}
\ollOption{restoration}{}

Used with \cmd{choice substitution}

Modification processes \emph{in the source}. \texttt{restoration} refers
to the case when a previously deleted text is restored to its original
state. It is recommended to use the \option{responsibility} and
\option{agent} attributes with these classes.

\ollOption{original}{}

Used with \cmd{choice normalization}

Refers to a musical text encoded literally although it deviates from the
desired presentation without being erroneous. Typical cases include the
distribution of hands to piano staves, abbreviations or similar
operations (both to music or text), stem (or other) directions, beaming
patterns etc.

\ollOption{regularization}{}

Documents that a text has been normalized or modernized in the sense of
the previous \option{original}.

\ollOption{gap}{}

This and the remaining classes are used together with
\cmd{choice emendation}.

Documents missing material in the source. Requires the attribute
\option{reason}.

\ollOption{sic}{}

Marks up erroneous content in the score.

\ollOption{unclear}{}

Used to mark up a text that can't be transcribed reliably. It is
encouraged to make use of the \option{certainty} and
\option{responsibility} attributes.

\ollOption{correction}{}

Encodes the text as corrected by the current editor. Requires the
\option{type} attribute, which must be one out of \option{addition},
\option{deletion}, and \option{substitution}. The use of
\option{certainty} and \option{responsibility} is encouraged.

\hypertarget{the-choice-module}{%
\section{\texorpdfstring{The \texttt{choice}
Module}{The choice Module}}\label{the-choice-module}}

In most cases one doesn't only want to ``tag'' musical elements with an
\cmd{editorialMarkup} but also their alternatives. When explicitly
naming corrections corrections one will often document the erroneous
text found in the source. This can add value to a textual commentary
describing the differences, and it can be used to create alternative
versions of an edition.

\cmd{choice} provides an infrastructure for this, enabling the encoding
of alternative versions of some music, choosing one version for use in
the engraving, and handling the annotations attached to the music spans.
Note that LilyPond can't support ``live'' updates to switch between
versions in real time.

\hypertarget{the-command-1}{%
\subsection{\texorpdfstring{The \cmd{choice}
Command}{The  Command}}\label{the-command-1}}

The \cmd{choice} command has the interface
\texttt{\textbackslash{}choice\ \textless{}choice-type\textgreater{}\ (\textless{}attributes\textgreater{})\ \textless{}music\textgreater{}}
where \texttt{\textless{}choice-type\textgreater{}} is an (partially
arbitrary) name, \texttt{\textless{}attributes\textgreater{}} and
optional \cmd{with \{\}} block with additional attributes, and music a
special type of music expression: it is a sequential music expression
whose child elements are \cmd{span} music expressions.
\cmd{editorialMarkup} fulfills that definition, too.

Choice types may be arbitrary but beyond four predefined types adding
custom types involves significant configuration and programming work.
These four types are tuned to work with \cmd{editorialCommand} in a
scholarly edition project: \texttt{variants}, \texttt{normalization},
\texttt{substitution}, and \texttt{emendation}. Each choice type has
rules regarding its children's span classes and a configurable function
for choosing the child to be used for engraving. The following example
documents that in the source one whole note has been changed into two
half notes. This is valid code, although in a real-world example the
finding would probably be described in more detail through attributes
like \option{reason}, \option{certainty} or \option{responsibility}. In
this case by default the ``new'' version is printed

\begin{Shaded}
\begin{Highlighting}[]

\end{Highlighting}
\end{Shaded}

\lilypondfile{choice-basic.ly}

\hypertarget{selecting-the-music-to-be-engraved}{%
\subsubsection{Selecting the Music to be
Engraved}\label{selecting-the-music-to-be-engraved}}

The selection which subexpression is used for engraving is controlled by
\texttt{preference} variables registered for each choice type. These are
set with \cmd{setChoicePreference <choice-type> <value>}.
\texttt{\textless{}choice-type\textgreater{}} is a Scheme symbol while
the value can be of arbitrary Scheme type, depending on the choice type.
The accepted values for the preconfigured choice types are documented
along with the types, but custom choice types may implement tests of
arbitrary complexity. Note that since arbitrary Scheme values are
accepted for the second argument also symbols have to be given in their
explicit Scheme notation with the prepended hash sign:
\cmd{setChoicePreference substitution \#'old}.

Generally, if the selection process fails (usually because there is no
suitable music expression available) the first encountered music
expression is chosen. Apart from that the order of encoding spans in a
choice expression is irrelevant.

\hypertarget{variants}{%
\subsubsection{\texorpdfstring{\texttt{variants}}{variants}}\label{variants}}

\cmd{choice variants} is used to encode alternative readings from
different sources. It must contain exactly one span of class
\option{lemma} and one or more \option{reading} span(s). Other span
classes are not allowed.

By default the \option{lemma} span is engraved, otherwise the
\texttt{preference} option must be set to the desired reading's
\option{source} attribute.

The following example sets up a \cmd{choice} as a music function. This
is unrealistically complicated but serves to show how changing the
preference uses different subexpressions from the choice:

\begin{Shaded}
\begin{Highlighting}[]

\end{Highlighting}
\end{Shaded}

\lilypondfile{choice-variants.ly}

\hypertarget{normalization}{%
\subsubsection{\texorpdfstring{\texttt{normalization}}{normalization}}\label{normalization}}

\cmd{choice normalization} is used to encode a version literally along
with some adaptation to modern or standardized editing conventions. The
expansion of abbreviations (e.g.~tremolos, repeats) also falls into this
category. This choice type is used when consistent or modern
presentation is desired and the deviations in the sources have to be
documented. Note that the original text is not considered faulty in this
case.

The choice must contain exactly one \option{original} and one
\option{regularization} span. By default the regularization is chosen,
and the preference values are \texttt{original} and
\texttt{regularization}.

\textbf{QUESTION:} \emph{Should they have a mandatory \texttt{type}
attribute to define things like type=abbreviation etc.?}

\hypertarget{substitution}{%
\subsubsection{\texorpdfstring{\texttt{substitution}}{substitution}}\label{substitution}}

\cmd{choice substitution} is used to document modifications applied
within the source, typically a correction from one text to a different
one.

The choice must contain one \option{deletion} and one \option{addition}
\emph{or} \option{restoration} span. By default the final state is
printed, and the preference values are \option{new} and \option{old} (as
Scheme symbols).

\hypertarget{emendation}{%
\subsubsection{\texorpdfstring{\texttt{emendation}}{emendation}}\label{emendation}}

\cmd{choice emendation} is used to document editorial decisions. The
choice must contain exactly two subexpressions, one being the
\option{correction}, the other being one out of \option{sic},
\option{gap} or \option{unclear}.

By default the correction is engraved, the preference values are
\option{old} and \option{new}.

\hypertarget{handling-annotations-in-a-choice-expression}{%
\subsection{Handling Annotations in a Choice
Expression}\label{handling-annotations-in-a-choice-expression}}

A central topic when dealing with choices of editorial markup
expressions is the handling of annotations. Details about annotations
can be looked up in the documentation of \ollPackage{stylesheets.span}
(annotations are created from a span expression) and the
\ollPackage{scholarly.annotate} chapter later in this document. But
\cmd{choice} provides some interesting features to manage annotations.

Any span (or editorial markup) expression implicitly carries a
\texttt{span-annotation} attached to its ``anchor'' element, and if that
includes an \option{ann-type} attribute a ``real'' annotation is created
and processed by \ollPackage{scholarly.annotate}. By design \cmd{choice}
selects one span expression from its subexpressions and returns that, so
implicitly the result of \cmd{choice} carries the annotation of the
chosen span.

If the \cmd{choice} itself has attributes too (the optional argument)
they are \emph{merged} with the chosen subexpression's annotation
attributes. If an attribute is present both in the choice and in the
selected subexpression the ``lower'' one from the subexpression
overwrites the one from the wrapping choice. This makes it possible to
create sophisticated annotations, for example to print an alternative
\texttt{message} text depending on the chosen subexpression.

\begin{Shaded}
\begin{Highlighting}[]

\end{Highlighting}
\end{Shaded}

\lilypondfile{choice-complex-annotation.ly}

\hypertarget{custom-choice-types}{%
\subsection{Custom Choice Types}\label{custom-choice-types}}

The \ollPackage{scholarly.choice} module has been developed with a
certain use case in mind and was therefore modeled after parts of the
MEI specification. However, nothing speaks against extending this with
custom choice types and rulesets.

Adding a custom choice type involves implementing and registering one
function for validating choice expressions and one for handling the
selection preference.

\hypertarget{creating-a-choice-validator}{%
\subsubsection{Creating a Choice
Validator}\label{creating-a-choice-validator}}

A choice validator is a function created with the macro
\option{(define-choice-validator)}. This creates a scheme-function
expecting one \texttt{choice-type} symbol and a \texttt{choice-music?}
music expression. The function body must consist of one expression
(optionally preceded by a docstring) and evaluate to a true value or
\texttt{\#f}.

Inside the function a number of variables and local functions are
available:

\begin{itemize}
\item
  \option{spans}

  a list of pairs with the span class as \texttt{car} and the music
  expression as \texttt{cdr}
\item
  \option{classes}

  an ordered list of span class names. \emph{Note:} generally the order
  of spans in a choice expressions is ignored, but it is \emph{possible}
  for custom validators to make a decision based on the order.
\item
  \option{(count-class <class>)}

  computes the number of times the given span class is present
\item
  \option{(single <class>)}

  returns \texttt{\#t} if the given class appears exactly once
\item
  \option{(optional <class>)}

  returns \texttt{\#c} if the given class appears zero or one times
\item
  \option{warning-message}

  if this variable is set to a string this string is output as part of
  the warning message reporting an invalid choice expression.
\end{itemize}

\hypertarget{registering-a-validator}{%
\subsubsection{Registering a Validator}\label{registering-a-validator}}

Once a validator function is defined it can be made accessible through
\cmd{setChoiceValidator <choice-type> <function>} where
\texttt{\textless{}choice-type\textgreater{}} is a Scheme symbol and
\texttt{\textless{}function\textgreater{}} a procedure. It is also
possible to register multiple validators at once with
\cmd{setChoiceValidators <validators-list}\} where the argument is an
association list linking choice type symbols to validator procedures.

\hypertarget{creating-a-span-chooser}{%
\subsubsection{Creating a Span Chooser}\label{creating-a-span-chooser}}

A span chooser is a function created with the macro
\option{(define-span-chooser)}. This creates a scheme-function expecting
one \texttt{choice-type} symbol, a \texttt{props} alist and a
\texttt{span-expressions?} list of span-class/span-music pairs. The
function body must consist of one expression (optionally preceded by a
docstring) and evaluate to a span-class/span-music pair.

Inside the function the following names are available:

\begin{itemize}
\tightlist
\item
  \texttt{preference}~\\
  a key as base data for the decision which span to choose. If the
  choice attributes include a \option{preference} attribute its value is
  taken, otherwise the value is looked up in the options as they are
  described above.
\item
  \texttt{(get-annotation\ \textless{}expression\textgreater{})}~\\
  a function to retrieve the span annotation from the given span
  expression. If the chooser function iterates over the expressions this
  is the way to access the current expression's annotation.
\end{itemize}

The built-in choosers all have comparably simple binary selection
mechanisms, but custom functions may implement conditions of arbitrary
complexity. Note that the \texttt{preference} value doesn't necessarily
have to be a simple symbol as in the built-in cases. it may also make
sense to use for example lists and choose the first expression that
happens to match a list element (use case: use readings from sources in
descending priority).

\hypertarget{registering-a-span-chooser}{%
\subsubsection{Registering a Span
Chooser}\label{registering-a-span-chooser}}

Once a span chooser is defined it can be made accessible through
\cmd{setSpanChooser <choice-type <function>} where
\texttt{\textless{}choice-type\textgreater{}} is a Scheme symbol and
\texttt{\textless{}function\textgreater{}} a procedure.

\hypertarget{the-sources-module}{%
\section{\texorpdfstring{The \texttt{sources}
Module}{The sources Module}}\label{the-sources-module}}

This is a stub as there is no \ollPackage{scholarly.sources} module yet,
not even a sketch. This module will provide an interface to storing
information about musical sources. Intended functionality:

\begin{itemize}
\tightlist
\item
  Reference by key in an annotation's \texttt{source} attribute
\item
  Produce output for source descriptions in critical reports\\
  (optionally: output only sources used (if possible/appropriate))
\item
  Maintain inheritance information to create a stemma.
\item
  Optionally: produce a graphical representation of a stemma (thorugh
  \LaTeX?)
\end{itemize}

\printindex
\addcontentsline{toc}{section}{Index}

\end{document}
