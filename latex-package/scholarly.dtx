% \iffalse meta-comment
%
% Copyright (C) 2016 by Jeffery Shivers <jefferyshivers@gmail.com>
% -------------------------------------------------------
%
% This file may be distributed and/or modified under the
% conditions of the LaTeX Project Public License, either version 1.2
% of this license or (at your option) any later version.
% The latest version of this license is in:
%
%    http://www.latex-project.org/lppl.txt
%
% and version 1.2 or later is part of all distributions of LaTeX
% version 1999/12/01 or later.
%
% \fi
%
% \iffalse
%<*driver>
\ProvidesFile{scholarly.dtx}
%</driver>
%<package>\NeedsTeXFormat{LaTeX2e}[1999/12/01]
%<package>\ProvidesPackage{scholarly}
%<*package>
    [2016/08/01 1.0 Package for typesetting annotations for critical editions
    compiled by LilyPond with the OpenLilyLib scholarLY library.]
%</package>
%
%<*driver>
\documentclass{ltxdoc}
\usepackage{idxlayout}
\usepackage[custom, final]{scholarly}
\usepackage{hypdoc}
\usepackage[titletoc]{appendix}
\usepackage{pbox}
\usepackage{xcolor}
\EnableCrossrefs
\CodelineIndex
\RecordChanges
\begin{document}
  \DocInput{scholarly.dtx}
\end{document}
%</driver>
% \fi
% \CheckSum{748}
% \CharacterTable
%  {Upper-case    \A\B\C\D\E\F\G\H\I\J\K\L\M\N\O\P\Q\R\S\T\U\V\W\X\Y\Z
%   Lower-case    \a\b\c\d\e\f\g\h\i\j\k\l\m\n\o\p\q\r\s\t\u\v\w\x\y\z
%   Digits        \0\1\2\3\4\5\6\7\8\9
%   Exclamation   \!     Double quote  \"     Hash (number) \#
%   Dollar        \$     Percent       \%     Ampersand     \&
%   Acute accent  \'     Left paren    \(     Right paren   \)
%   Asterisk      \*     Plus          \+     Comma         \,
%   Minus         \-     Point         \.     Solidus       \/
%   Colon         \:     Semicolon     \;     Less than     \<
%   Equals        \=     Greater than  \>     Question mark \?
%   Commercial at \@     Left bracket  \[     Backslash     \\
%   Right bracket \]     Circumflex    \^     Underscore    \_
%   Grave accent  \`     Left brace    \{     Vertical bar  \|
%   Right brace   \}     Tilde         \~}
% \changes{v1.0}{2016/08/01}{Initial version}
% \GetFileInfo{scholarly.dtx}
% \DoNotIndex{\newcommand,\newenvironment,\def,\gdef,\edef}
%
% \def\scholarLY{\textsf{scholarLY} }
% \def\scholarLYs{\textsf{scholarLY}'s }
%
%
%
% \title{The \scholarLY package\thanks{This document
%   corresponds to \textsf{scholarLY}~\fileversion, dated \filedate.
%   The package is available online at
%   \url{http://www.ctan.org} and
%   \url{https://github.com/openlilylib/scholarly}.}\\
%   version \fileversion}
% \author{Jeffery Shivers \\ \texttt{jefferyshivers@gmail.com}}
% \date{\filedate}
% \maketitle
%
% \definecolor{blueberry}{rgb}{0.1,0.1,0.3}
%
% {\hypersetup{linkcolor=blueberry}
% \tableofcontents}
%
% \newpage
% \section{Introduction}
%
% The aim of the \scholarLY package is to provide the tools
% for typesetting annotations for critical editions of musical scores
% engraved with GNU LilyPond.
% This \LaTeX{} package processes the |<filename>.annotations.inp| file
% exported from LilyPond using the openLilyLib \scholarLY library. The
% main features of the package currently include:
%
% \medskip
% \begin{itemize}[nosep]
% \item{formatting and styling annotation parameters individually and dynamically / inline}
% \item{package options for toggling |final| or |draft| and |custom| or |default| modes}
% \item{simple key-value infrastructure for organizing style blocks}
% \end{itemize}
% \medskip
%
% In this initial release, \scholarLY is a relatively
% lightweight package. It is foremost concerned with the typsetting of
% annotations, and as such does not provide a means of directly
% interacting with musical scores, nor with \LaTeX{} documents outside
% of the annotations themselves.  Future minor releases may introduce
% extended capabilities, particularly ones to better facilitate the
% integration of various compilation
% procedures\footnote{\url{http://lilypond.org/doc/v2.19/Documentation/usage/integrating-music-and-text}}
% (such as with |lilypond-book| and the \textsf{musicexamples}
% package).
%
% \bigskip
% \hrule
% \bigskip
%
% Feature requests, bug reports, and other concerns should be directed
% to the \scholarLY issue tracker at
% \url{https://github.com/openlilylib/scholarly/issues}. That is also
% the best location to observe the latest state of the project, and to
% explore the rest of the scholarLY project and openLilyLib's vast and
% growing resources.
%
% \section{Usage}
%
% \newcommand{\exampleAnn}[1]
%     {\begEnumerate\annStyleMode\annRevisit{#1}\endEnumerate}
%
%     It is possible to simply print our annotations with a single
%     command in the document. In fact, it is
%     useful to do so before applying any customizations in order
%     to ensure all annotations are being correctly received. To add
%     customized styles to the annotations, \scholarLY style blocks provide a
%     cleanly nested infastructure for organizing behaviors of
%     specific details in each mode.  Assuming the aim is to combine
%     the annotations and the musical score into a single document,
%     this basic functionality gets only a little more
%     complicated. This documentation will not go through the various
%     ways to compile LilyPond scores with \LaTeX{}
%     documents,\footnote{\url{http://lilypond.org/doc/v2.19/Documentation/usage/latex.en.html}}
%     but hopefully it will become clear that \scholarLYs
%     flexibility is suited for a range of approaches.
%
% \subsection{The Main Command}
%
% At the center of the \scholarLY package is the main |\annotations|
% command, which wraps all of our presets and annotations with a
% single argument. It processes the |.inp| file which was exported
% from the corresponding LilyPond document.\footnote{This
% documentation doesn't provide instructions for using the LilyPond
% side of \textsf{scholarLY}; for that, please go to
% \url{https://github.com/openlilylib/scholarly} and view the README,
% as well as the |annotate.ly|, |footnotes.ly| and
% |editorial-commands.ly| example documents.}
%
% \begin{verbatim}
%    \annotations{myscore.annotations.inp}
% \end{verbatim}
%
% This hook will apply any predefined settings to the list(s). A
% single document can contain multiple |\annotations| instances as
% long as the named files are in \LaTeX 's search path.  It does not
% take optional arguments (yet), so all settings must be instructed
% either in the |\usepackage| message or within \scholarLY
% style blocks.
%
% \subsection{Top-Level Options}
% In the |\usepackage| statement, we can make use of two global options.
% \begin{verbatim}
%    \usepackage[custom,final]{scholarLY}
% \end{verbatim}
% \begin{description}[noitemsep]
% \item[draft / final:] This option offers two concurrent modes for
%   storing custom styles. Selecting either one will tell
%   \scholarLY to apply the respective settings to the document. The
%   names \textit{final} and
%   \textit{draft} are more or less arbitrary, but they are hard-coded
%   into the package, so \scholarLY expects one of the
%   two. |final| is the implicit option if neither is specified.
% \item[default / custom:] This option toggles between the default
%   styles of \scholarLY and the user-defined options. It
%   offers an easy way to turn on or off your custom settings quickly without
%   actually changing anything in the document. The default setting is
%   |custom| (if no styles are specified, \scholarLY will
%   still print your annotations without any settings; they'll just be
%   very bland).
% \end{description}
%
% As described above, the implicit setting is |[custom,final]|.
% However, it's best to be verbose about the intended options, as in the
% example, in order to remove any ambiguity or confusion later on.
%
% This package makes use of a default
% stylesheet\footnote{|default-stylesheet.inp| is located in the same
% directory as the package file.} for its |default| mode, which
% inherently contains almost all of the available commands, and so it
% doubles as a convenient example of how style blocks could be
% organized. When |default| mode is turned on, \scholarLY will
% revert all styles and formats to whatever is listed in that
% document, according to either final or draft mode. The biggest
% differences between |default|'s |final| and |draft| modes are that
% |draft| will print annotation parameters stacked, show annotation
% types and enumerate the list.  An annotation in |[default,draft]|
% mode will appear as:
%
% \bigskip
%
% \noindent\fcolorbox{gray}{white}{\parbox{\linewidth}{
%
% \annShowType\annStacked
% \exampleAnn{
%   \criticalRemark
%   [grob-location={((beat-string . 4) (beat-fraction . 0) (beat-part . Mom 0)
%   (our-beat . 4) (measure-pos . Mom 3/4) (measure-no . 1) (rhythmic-location 1 . Mom 3/4))},
%   grob-type={Accidental},
%   type={critical-remark},
%   context-id={Viola},
%   message={This note should be sharp.}]
% }
% }}
% \bigskip
%
% This is of course not very interesting, typographically
% speaking. That is intentional, as |default| mode was created to be a
% stylistically simple interface for overviewing annotations and
% potential errors, etc. therein.  The default stylesheet can be
% altered, but it is recommended to leave it fairly minimal so that
% you'll have an easy way to revert settings, which could be
% occasionally useful for proof-reading or debugging the document.
%
% \subsection{Style Blocks}
%
% We can contain all of our custom settings in two style blocks. This
% allows \scholarLY to toggle between |draft| and |final|
% settings easily, and of course provides a clean syntax for
% organizing the code.
%
% To set custom |draft| and |final| behaviors, we use:
% \begin{verbatim}
%    \annSetDraft{
%      [..]
%    }
%    \annSetFinal{
%      [..]
%    }
% \end{verbatim}
%
% \scholarLY allows for unlimited |\annSet| blocks. Note,
% however, that creating a new block doesn't automatically override
% all items set in previous blocks until they are explicitly
% redefined.\footnote{Hint: the \textsf{verbatim} package provides a
% convenient environment for commenting out code blocks. This could be
% one tool for quickly and easily checking the behavior of your custom
% styles.}  It is unlikely that most documents will have a need to use
% very many |\annotations| instances and style preferences, so that is
% a more or less trivial detail.
%
% Inside the style blocks, we can insert any important hooks such as
% |\annStacked| and |\annShowType|, as well as three key-value groups,
% using \textsf{keyval} package's |\setkeys| macro: |annStyles|,
% |annPrefixes| and |annExtras|. Here is an example of a custom style
% block followed by an example of a resulting annotation in |final|
% mode:
%
% \begin{minipage}{\linewidth}
% \begin{verbatim}
%    \annSetFinal{
%      \setkeys{annStyles}{
%        message = {\color{cyan}},
%        message-2nd = {\textit},
%        affected = {\MakeLowercase},
%        voice = {\textbf},
%        voice-2nd = {\MakeLowercase}
%      }
%      \setkeys{annPrefixes}{
%        measure = {m.},
%        beat = {b.}
%      }
%      \annHideType
%      \annInline
%      \annEnumerateOff
%    }
% \end{verbatim}
% \end{minipage}
%
% \annSetFinal{
%    \setkeys{annStyles}{
%      message = {\color{cyan}},
%      message-2nd = {\textit},
%      affected = {\MakeLowercase},
%      voice = {\textbf},
%      voice-2nd = {\MakeLowercase}
%    }
%  \annInline
% }
%  \annHideType
%  \annEnumerateOff
%
% \noindent\fcolorbox{gray}{white}{\parbox{\linewidth}{
%
% \exampleAnn{
% \criticalRemark
%   [grob-location={((beat-string . 4) (beat-fraction . 0) (beat-part . Mom 0)
%   (our-beat . 4) (measure-pos . Mom 3/4) (measure-no . 1) (rhythmic-location 1 . Mom 3/4))},
%   grob-type={Accidental},
%   type={critical-remark},
%   context-id={Viola},
%   message={This note should be sharp.}]
% }
% }}
%
% \bigskip
%
%   Note the usage of |message| and |message-2nd|, and |voice| and
%   |voice-2nd| in the |annStyles| settings. Since some macros will
%   likely need to enclose the affected expressions, it is sometimes
%   necessary for each macro after the first to also wrap the
%   preceding \textit{inner} arguments. In effect, they are applied
%   as:
%
%   \bigskip { \begin{center} \{ macro-3rd \{
%     macro-2nd \{ macro \{ item \} \} \} \} \end{center} } \bigskip
%
%   Annotations also adhere to other style commands that might be
%   established in the \LaTeX{} document.  Font faces, spacing,
%   etc. will be adapted if previously set (which you'll presumably
%   want). We can wrap |\annotations| in explicit \LaTeX{}
%   style commands just the same as with other parts of the
%   document. Keep in mind, however, that choices made in
%   |\annSet| blocks will override any other conflicting styles
%   since they are being applied at the instance of the |\annotations| hook,
%   which means it will normally expand \textit{after} any macros or environments
%   in which it might be wrapped.\footnote{Of course, there are tricks
%   around everything: to cancel your own style blocks and
%   \textit{not} use the default stylesheet, simply follow your custom
%   blocks with the same respective blocks \textit{emptied}. Since
%   there is no convenience command for this, you'll have to manually
%   type each of the options you wish to set to nothing.} It is
%   recommended to maintain annotation settings within style blocks in
%   order to keep |draft| and |final| modes as clearly defined as
%   possible.
%
% \subsection{More Examples}
%
% Here are some additional examples to demonstrate how annotations might be
% formatted.
%
%
%
%
%
%
% \subsubsection{Colors, Emphasis}
%
% \begin{verbatim}
% \annSetFinal{
%   \setkeys{annStyles}{
%     message={\textbf},
%     type={\small\textit\color{red}}
%   }
% }
% \end{verbatim}
%
% \noindent\fcolorbox{gray}{white}{\parbox{\linewidth}{
%
% \annSetFinal{
%   \setkeys{annStyles}{
%     message={\textbf},
%     type={\color{red}\textit}
%   }
%   \setkeys{annPrefixes}{
%     beat={b.}
%   }
%   \annMessageWrap{``/''}
%   \annEnumerateOff
%   \annShowType
%   \annInline
%   \annTypeWrap{(/) }
% }
%
% \exampleAnn{
% \criticalRemark
%   [grob-location={((beat-string . 4) (beat-fraction . 0) (beat-part . Mom 0)
%   (our-beat . 4) (measure-pos . Mom 3/4) (measure-no . 1) (rhythmic-location 1 . Mom 3/4))},
%   grob-type={Accidental},
%   type={todo},
%   context-id={Viola},
%   message={Verify this pitch.}]
% }
% }}
%
%
%
%
%
%
% \subsubsection{Prefixes, Other Text Styles}
%
% \begin{verbatim}
% \annSetFinal{
%   \setkeys{annStyles}{
%     message={\large}
%   }
%   \setkeys{annPrefixes}{
%     beat={b.},
%     location={\textit{\color{darkgray}location: }},
%     context={\textit{\color{darkgray}context: }},
%     grob-type={\textit{\color{darkgray}affects: }}
%   }
%   \annMessageWrap{/}
%   \annStacked
% }
% \annHideType
% \annEnumerateOn
% \end{verbatim}
%
% \annSetFinal{
%   \setkeys{annStyles}{
%     message={\large}
%   }
%   \setkeys{annPrefixes}{
%     beat={b.},
%     location={\textit{\color{darkgray}location: }},
%     context={\textit{\color{darkgray}part: }},
%     grob-type={\textit{\color{darkgray}affects: }}
%   }
%   \annMessageWrap{/}
%   \annStacked
%   \renewcommand\labelenumi{(\theenumi)}
% }
%   \annHideType
%   \annEnumerateOn
%
% \noindent\fcolorbox{gray}{white}{\parbox{\linewidth}{
% \exampleAnn{
% \criticalRemark
%   [grob-location={((beat-string . 4) (beat-fraction . 0) (beat-part . Mom 0)
%   (our-beat . 4) (measure-pos . Mom 3/4) (measure-no . 1) (rhythmic-location 1 . Mom 3/4))},
%   grob-type={Accidental},
%   type={critical-remark},
%   context-id={Viola},
%   message={This note should be sharp.}]
% }
% }}
%
% \subsubsection{ex. 3}
%
% \noindent\fcolorbox{gray}{white}{\parbox{\linewidth}{
%
% \exampleAnn{
% \criticalRemark
%   [grob-location={((beat-string . 4) (beat-fraction . 0) (beat-part . Mom 0)
%   (our-beat . 3) (measure-pos . Mom 3/4) (measure-no . 1) (rhythmic-location 1 . Mom 3/4))},
%   grob-type={Accidental},
%   type={critical-remark},
%   context-id={Viola},
%   message={This note should be sharp.}]
%
% \criticalRemark
%   [grob-location={((beat-string . 4) (beat-fraction . 0) (beat-part . Mom 0)
%   (our-beat . 4) (measure-pos . Mom 3/4) (measure-no . 1) (rhythmic-location 1 . Mom 3/4))},
%   grob-type={NoteHead},
%   type={todo},
%   context-id={Violoncello},
%   message={I should reconsider this note.}]
%
% \criticalRemark
%   [grob-location={((beat-string . 4) (beat-fraction . 0) (beat-part . Mom 0)
%   (our-beat . 4) (measure-pos . Mom 3/4) (measure-no . 1) (rhythmic-location 1 . Mom 3/4))},
%   grob-type={Slur},
%   type={lilypond-issue},
%   context-id={Double Bass},
%   message={This slur looks bad.}]
% }
%
% }}
%
% \begin{verbatim}
% \annSetFinal{
%   \annSameLocationStrict
% }
% \end{verbatim}
%
% \annSameLocationStrict
%
% \bigskip
% \noindent\fcolorbox{gray}{white}{\parbox{\linewidth}{
%
% \exampleAnn{
% \criticalRemark
%   [grob-location={((beat-string . 4) (beat-fraction . 0) (beat-part . Mom 0)
%   (our-beat . 3) (measure-pos . Mom 3/4) (measure-no . 1) (rhythmic-location 1 . Mom 3/4))},
%   grob-type={Accidental},
%   type={critical-remark},
%   context-id={Viola},
%   message={This note should be sharp.}]
%
% \criticalRemark
%   [grob-location={((beat-string . 4) (beat-fraction . 0) (beat-part . Mom 0)
%   (our-beat . 4) (measure-pos . Mom 3/4) (measure-no . 1) (rhythmic-location 1 . Mom 3/4))},
%   grob-type={Accidental},
%   type={critical-remark},
%   context-id={Viola},
%   message={This note should be sharp.}]
%
% \criticalRemark
%   [grob-location={((beat-string . 4) (beat-fraction . 0) (beat-part . Mom 0)
%   (our-beat . 4) (measure-pos . Mom 3/4) (measure-no . 1) (rhythmic-location 1 . Mom 3/4))},
%   grob-type={Accidental},
%   type={critical-remark},
%   context-id={Viola},
%   message={This note should be sharp.}]
% }
%
% }}
%
%   \bigskip
%
% |code|
%
% \bigskip
% \noindent\fcolorbox{gray}{white}{\parbox{\linewidth}{
%
% \setkeys{annPrefixes}{
%   post-type = {\vspace{-0.5em}\\{\color{red}\hrule}\vspace{0.5em}}
% }
%   \annShowType
%   \annTypeWrap{(/)}
%   \annEnumerateOn
% \annStacked
%
% \exampleAnn{
% \criticalRemark
%   [grob-location={((beat-string . 4) (beat-fraction . 0) (beat-part . Mom 0)
%   (our-beat . 4) (measure-pos . Mom 3/4) (measure-no . 1) (rhythmic-location 1 . Mom 3/4))},
%   grob-type={Accidental},
%   type={critical-remark},
%   context-id={Viola},
%   message={This note should be sharp.}]
%
% \criticalRemark
%   [grob-location={((beat-string . 4) (beat-fraction . 0) (beat-part . Mom 0)
%   (our-beat . 4) (measure-pos . Mom 3/4) (measure-no . 3) (rhythmic-location 1 . Mom 3/4))},
%   grob-type={Accidental},
%   type={critical-remark},
%   context-id={Viola},
%   message={This note should be sharp.}]
% }
% }}
%
%
%
%
%
% \clearpage
%
% \StopEventually{}
%\iffalse
%<*package>
%\fi
%
% \section{Implementation}
%
% The majority of this package corresponds to the default annotation
% parser.  In order to use a custom annotation parser, I suggest
% looking closely at section 2.6 in particular to get an idea of what
% functions are already available. Because of the inherent
% unpredictability of how custom parsers will be constructed and thus
% behave, \textsf{scholarLY} doesn't force location checks, key
% resets, and other functionality that users will most likely need. At
% this point, the use of custom annotation parsing isn't fully
% supported (i.e. the bare mechanisms are in place (section 2.7), but
% have not really been tested yet). With that said, here are the
% current leading TODO's:
%
% \medskip
% \begin{itemize}[nosep]
% \item{Fully support custom annotation parsers.}
% \item{Address beat fractions; currently scholarLY only supports whole beat locations.}
% \item{Rewrite custom footnote parser; it should be a loop that allows for infinite footnotes,
%   but currently only allows up to five, plus one annotation footnote.}
% \item{Add automatic labels to annotations for referencing / linking.}
% \end{itemize}
% \medskip
%
% \subsection{Required Packages}
%
%    \begin{macrocode}
\RequirePackage{ifthen}
\RequirePackage{enumitem}
\RequirePackage{xstring}
\RequirePackage{keyval}
\RequirePackage{titlecaps}
%    \end{macrocode}
%
% \subsection{Style Annotations}
%
% Set basic styles with |\setkeys{annStyles}{ .. }|
%    \begin{macrocode}
\def\annName{}
\def\annType{}
\def\annMeasure{}
\def\annBeat{}
\def\annVoice{}
\def\annVoiceO{}
\def\annAffected{}
\def\annAffectedO{}
\def\annMessage{}
\def\annMessageO{}
\def\annMessageOO{}
\define@key{annStyles}{type}{\def\annType{#1}}
\define@key{annStyles}{measure}{\def\annMeasure{#1}}
\define@key{annStyles}{beat}{\def\annBeat{#1}}
\define@key{annStyles}{voice}{\def\annVoice{#1}}
\define@key{annStyles}{voice-2nd}{\def\annVoiceO{#1}}
\define@key{annStyles}{affected}{\def\annAffected{#1}}
\define@key{annStyles}{affected-2nd}{\def\annAffectedO{#1}}
\define@key{annStyles}{message}{\def\annMessage{#1}}
\define@key{annStyles}{message-2nd}{\def\annMessageO{#1}}
\define@key{annStyles}{message-3rd}{\def\annMessageOO{#1}}
%    \end{macrocode}
%
%
% draft and final mode (used internally only)
%    \begin{macrocode}
\def\customdraft{}
\newcommand\annSetDraft[1]{
  \def\customdraft{#1}}
\def\customfinal{}
\newcommand\annSetFinal[1]{
  \def\customfinal{#1}}
%    \end{macrocode}
%
% \begin{macro}{\finalfordraft}
%   Let custom final settings carry over to draft mode.\footnote{footnote}
% This feature might be removed if it proves to be excessive.
%    \begin{macrocode}
\newcommand{\finalfordraft}{\def\customdraft{\customfinal}}
%    \end{macrocode}
% \end{macro}
%    \begin{macrocode}
\newcommand{\letAnnBreakornot}{\par\nobreak}
%    \end{macrocode}
% \begin{macro}{\annBreakAllow}
% Let annotations break over pages/columns.
%    \begin{macrocode}
\newcommand{\annBreakAllow}
  {\renewcommand{\letAnnBreakornot}{\par}}
%    \end{macrocode}
% \end{macro}
% \begin{macro}{\annBreakAvoid}
% Avoid breaks over pages/columns.
%    \begin{macrocode}
\newcommand{\annBreakAvoid}
  {\renewcommand{\letAnnBreakornot}{\par\nobreak}}
%    \end{macrocode}
% \end{macro}
% \begin{macro}{\setEnumOptions}
% Apply any of the usual |enumerate| options to the list; takes one argument, which may itself contain multiple commands.
%    \begin{macrocode}
\newcommand{\enumOptions}{}
\newcommand{\setEnumOptions}[1]{\renewcommand{\enumOptions}{#1}}
%    \end{macrocode}
%    \begin{macrocode}
\newcommand{\skipornot}{}
\newcommand{\typeSkipornot}{}
%    \end{macrocode}
% \end{macro}
% \begin{macro}{\annInline}
% Print annotations with inline parameters.
%    \begin{macrocode}
\newcommand{\annInline}{%
  \renewcommand{\skipornot}{, }
  \renewcommand{\typeSkipornot}{}}
%    \end{macrocode}
% \end{macro}
% \begin{macro}{\annStacked}
% Print annotations with stacked parameters.
%    \begin{macrocode}
\newcommand{\annStacked}{%
  \renewcommand{\skipornot}{\letAnnBreakornot}
  \renewcommand{\typeSkipornot}{\letAnnBreakornot}}
%    \end{macrocode}
% \end{macro}
% custom type name
%    \begin{macrocode}
\define@key{scholarLYann}{type}{\def\lytype{#1}}
\def\prependType{}
\def\appendType{}
\def\annType{}
%    \end{macrocode}
% \begin{macro}{\annHideType}
% hide annotation type:
%    \begin{macrocode}
\newcommand{\annHideType}{%
  \def\annName{}
  \renewcommand{\typeSkipornot}{}}
%    \end{macrocode}
% \end{macro}
% \begin{macro}{\annHideType}
% hide annotation type:
%    \begin{macrocode}
\newcommand{\annShowType}{%
  \def\annName{{\annType\annTypeName }}
  \ifthenelse{\equal{\skipornot}{\letAnnBreakornot}}
    {\renewcommand{\typeSkipornot}{\letAnnBreakornot}}
    {\renewcommand{\typeSkipornot}{}}}
%    \end{macrocode}
% \end{macro}
% insert the default stylesheet used by `default` mode
%    \begin{macrocode}
\input{default-stylesheet.inp}
\def\annStyleMode{}
\def\annRevisit{}
\def\revisitFinalorNot{}
\def\revisitDraftorNot{}
%    \end{macrocode}
%
% \subsection{Package Options}
%
% implicitly |[custom, final]|
% \begin{macro}{draft}
%    \begin{macrocode}
\DeclareOption{draft}{
  \def\annStyleMode{\customdraft}
  \def\annRevisit{\revisitDraftorNot}}
%    \end{macrocode}
% \end{macro}
% \begin{macro}{final}
%    \begin{macrocode}
\DeclareOption{final}{
  \def\annStyleMode{\customfinal}
  \def\annRevisit{\revisitFinalorNot}}
%    \end{macrocode}
% \end{macro}
% \begin{macro}{default}
%    \begin{macrocode}
\DeclareOption{default}{
  \def\revisitFinalorNot{\defaultFinal}
  \def\revisitDraftorNot{\defaultDraft}}
%    \end{macrocode}
% \end{macro}
% \begin{macro}{custom}
%    \begin{macrocode}
\DeclareOption{custom}{
  \def\revisitFinalorNot{}
  \def\revisitDraftorNot{}}
\ExecuteOptions{custom,final}
\ProcessOptions\relax
%    \end{macrocode}
% \end{macro}
%
% \subsection{Additional Customizations}
%
% do something immediately after the type (only effective if type present)
%    \begin{macrocode}
\def\annPostType{}
\define@key{annPrefixes}{post-type}{\def\annPostType{#1}}
%    \end{macrocode}
% do something immediately before each type or arg (arg presently affects all):
%    \begin{macrocode}
\def\annTypePrePrefix{}
\define@key{annPrefixes}{pre-type}{\def\annTypePrePrefix{#1}}
\def\annArgsPrePrefix{}
\define@key{annPrefixes}{pre-args}{\def\annArgsPrePrefix{#1}}
%    \end{macrocode}
% first message wrapper (goes inside ann-footnote)
%    \begin{macrocode}
\def\prependMessage{}
\def\appendMessage{}
\def\aMWSep{/}
%    \end{macrocode}
% macro to redefine the seperator
%    \begin{macrocode}
\newcommand\annMessageWrapSep[1]{\def\aMWSep{#1}}
%    \end{macrocode}
% macro to define message wrap
%    \begin{macrocode}
\newcommand\annMessageWrap[1]{
  \expandarg\IfSubStr{#1}{\aMWSep}
    {\StrCut{#1}{\aMWSep}\prependMessage\appendMessage}
    {\def\prependMessage{#1}
    \def\appendMessage{#1}}}
%    \end{macrocode}
% outer message wrap
%    \begin{macrocode}
\def\prependMessageFirst{}
\def\appendMessageLast{}
\def\aMWOSep{/}
%    \end{macrocode}
% macro to redefine the seperator
%    \begin{macrocode}
\newcommand\annMessageWrapOuterSep[1]{\def\aMWOSep{#1}}
%    \end{macrocode}
% macro to define outer message wrap
%    \begin{macrocode}
\newcommand\annMessageWrapOuter[1]{
  \expandarg\IfSubStr{#1}{\aMWOSep}
    {\StrCut{#1}{\aMWOSep}\prependMessageFirst\appendMessageLast}
    {\def\prependMessageFirst{#1}
    \def\appendMessageLast{#1}}}
%    \end{macrocode}
% ann type wrapper
%    \begin{macrocode}
\def\prependType{}
\def\appendType{}
\def\aTWSep{/}
%    \end{macrocode}
% macro to redefine the seperator
%    \begin{macrocode}
\newcommand\annTypeWrapSep[1]{\def\aTWSep{#1}}
%    \end{macrocode}
% macro to define type wrap
%    \begin{macrocode}
\newcommand\annTypeWrap[1]{
  \expandarg\IfSubStr{#1}{\aTWSep}
    {\StrCut{#1}{\aTWSep}\prependType\appendType}
    {\def\prependType{#1}
    \def\appendType{#1}}}
%    \end{macrocode}
% enumerate or not, using enumitem package
%    \begin{macrocode}
\def\annEnumerateOn{
  \def\itemornot{\item}
  \def\begEnumerate{\begin{enumerate}[mode=boxed]\enumOptions}
  \def\endEnumerate{\end{enumerate}}}
\def\annEnumerateOff{
  \def\itemornot{}
  \def\begEnumerate{}
  \def\endEnumerate{}}
\annEnumerateOn
%    \end{macrocode}
% main annotation command
%    \begin{macrocode}
\newcommand{\annotations}[1]
{\begEnumerate\annStyleMode\annRevisit\input{#1}\endEnumerate}
%    \end{macrocode}
% keys originally set in arg 1, exported from lilypond
%    \begin{macrocode}
\define@key{scholarLYann}{grob}{\def\lygrob{#1}}
\define@key{scholarLYann}{grob-location}{\def\lygroblocation{#1}}
\define@key{scholarLYann}{grob-type}{\def\lygrobtype{#1}}
\define@key{scholarLYann}{input-file-name}{\def\lyinputfilename{#1}}
\define@key{scholarLYann}{context-id}{\def\lycontextid{#1}}
\define@key{scholarLYann}{location}{\def\lylocation{#1}}
\define@key{scholarLYann}{message}{\def\lymessage{#1}}
\define@key{scholarLYann}{type}{\def\lyanntype{#1}}
\define@key{scholarLYann}{context}{\def\lycontext{#1}}
\define@key{scholarLYann}{ann-footnote}{\def\lyannfootnote{#1}}
\def\resetkeys{
  \setkeys{scholarLYann}{%
    grob = no value given,
    grob-location = no value given,
    grob-type = no value given,
    input-file-name = no value given,
    context-id = no value given,
    location = no value given,
    type = no value given,
    message = no value given,
    context = no value given,
    ann-footnote = {no value given}
  }
}
\def\noMessError{\color{red}
    Oops! No message has been entered for this annotation.}
%    \end{macrocode}
% default Prefixes
%    \begin{macrocode}
\def\annGrobPrx{}
\def\annGrobLocationPfx{}
\def\annGrobTypePfx{}
\def\lyinputfilenamePfx{}
\def\lycontextidPfx{}
\def\annLocationPfx{}
\def\annTypePfx{}
\def\annMessagePfx{}
\def\annContextPfx{}
\def\annMeasurePfx{M.}
\def\annBeatPfx{beat }
\define@key{annPrefixes}{grob}{\def\annGrobPfx{#1}}
\define@key{annPrefixes}{grob-location}{\def\annGrobLocationPfx{#1}}
\define@key{annPrefixes}{grob-type}{\def\annGrobTypePfx{#1}}
\define@key{annPrefixes}{input-file-name}{\def\annInputFileNamePfx{#1}}
\define@key{annPrefixes}{context-id}{\def\annContextIDPfx{#1}}
\define@key{annPrefixes}{location}{\def\annLocationPfx{#1}}
\define@key{annPrefixes}{type}{\def\annTypePfx{#1}}
\define@key{annPrefixes}{message}{\def\annMessagePfx{#1}}
\define@key{annPrefixes}{context}{\def\annContextPfx{#1}}
\define@key{annPrefixes}{measure}{\def\annMeasurePfx{#1}}
\define@key{annPrefixes}{beat}{\def\annBeatPfx{#1}}
%    \end{macrocode}
% retrieve measure and beat from grob-location (doesn't support complex fractions yet)
%    \begin{macrocode}
\def\processGrobLocation{
  \StrBetween[1,1]{\lygroblocation}
    {(beat-string .}{)}[\lybeatstring]
    \StrDel{\lybeatstring}{ }[\lybeatstring]
  \StrBetween[1,2]{\lygroblocation}
    {(beat-fraction .}{)}[\lybeatfraction]
    \StrDel{\lybeatfraction}{ }[\lybeatfraction]
  \StrBetween[1,3]{\lygroblocation}
    {(beat-part .}{)}[\lybeatpart]
    \StrDel{\lybeatpart}{ }[\lybeatpart]
  \StrBetween[1,4]{\lygroblocation}
    {(our-beat .}{)}[\lyourbeat]
    \StrDel{\lyourbeat}{ }[\lyourbeat]
    \StrDel{\lyourbeat}{ }[\lyourbeatPrev]
  \StrBetween[1,5]{\lygroblocation}
    {(measure-pos .}{)}[\lymeasurepos]
    \StrDel{\lymeasurepos}{ }[\lymeasurepos]
  \StrBetween[1,6]{\lygroblocation}
    {(measure-no .}{)}[\lymeasureno]
    \StrDel{\lymeasureno}{ }[\lymeasureno]
  \StrBetween[1,7]{\lygroblocation}
    {(rhythmic-location }{)}[\lyrhythmiclocation]}
%    \end{macrocode}
% if same measure, optionally substitute:
%    \begin{macrocode}
\def\previousMeasure{0}
\def\currentMeasure{0}
\def\sameMeasureStub{---}
\define@key{annExtras}{same-measure}{\def\sameMeasureStub{#1}}
%    \end{macrocode}
% optionally substitute same beat (if *also* same measure):
%    \begin{macrocode}
\def\previousBeat{0}
\def\currentBeat{0}
\def\sameBeatStub{{---}\unskip}
\define@key{annExtras}{same-beat}{\def\sameBeatStub{#1}}
%    \end{macrocode}
% \begin{macro}{\annSameLocationLoose}
% to sub same (measure, or measure+beat (never just beat)) when same:
%    \begin{macrocode}
\def\sameLocationLoose{
  \ifthenelse{\equal{\currentBeat}{\previousBeat}}
    {\ifthenelse{\equal{\currentMeasure}{\previousMeasure}}
      {\def\thisMeasure{\sameMeasureStub}
      \def\thisBeat{}}
      {\def\thisMeasure{\annMeasurePfx\currentMeasure,}
      \def\thisBeat{\annBeatPfx\currentBeat}}}
    {\ifthenelse{\equal{\currentMeasure}{\previousMeasure}}
        {\def\thisMeasure{\sameMeasureStub}
        \def\thisBeat{\annBeatPfx\currentBeat}}
        {\def\thisMeasure{\annMeasurePfx\currentMeasure,}
        \def\thisBeat{\annBeatPfx\currentBeat}}}}
%    \end{macrocode}
% \end{macro}
% \begin{macro}{\annSameLocationStrict}
% to sub same ONLY when both same:
%    \begin{macrocode}
\def\sameLocationStrict{
  \ifthenelse{\equal{\currentMeasure}{\previousMeasure}}
    {\ifthenelse{\equal{\currentBeat}{\previousBeat}}
      {\def\thisMeasure{\sameMeasureStub}
      \def\thisBeat{\unskip}}
      {\def\thisMeasure{\annMeasurePfx\currentMeasure,}
      \def\thisBeat{\annBeatPfx\currentBeat}}}
    {\def\thisMeasure{\annMeasurePfx\currentMeasure,}
    \def\thisBeat{\annBeatPfx\currentBeat}}}
\def\annSameLocationStrict{\def\testLocation{\sameLocationStrict}}
%    \end{macrocode}
% \end{macro}
% \begin{macro}{\annSameLocationShow}
% to always show both
%    \begin{macrocode}
\def\sameLocationShow{
  \def\thisMeasure{\annMeasurePfx\currentMeasure,}
  \def\thisBeat{\annBeatPfx\currentBeat}}
%    \end{macrocode}
% \end{macro}
% macros for show / strict-hide / loose-hide
%    \begin{macrocode}
\def\annSameLocationShow{\def\testLocation{\sameLocationShow}}
\def\annSameLocationLoose{\def\testLocation{\sameLocationLoose}}
\annSameLocationLoose
%    \end{macrocode}
%
% \subsection{Embedded Footnotes}
%
% This functionality is a TODO. Once implemented, it will provide the
% ability to use custom footnote hooks within the annotation \textsf{message}
% in LilyPond and to define them (i.e. type the actual footnote text) there as
% separate properties.
%
% For now, and at any point onward, footnotes can of course still be explicity
% typed into the annotation message with the traditional \textsf{footnote}
% hook. There is additionally the option to define a footnote for the full
% annotation, which guarantees that the footnote superscript will be placed
% at the end of the full annotation (including punctuation).
%
% \subsection{Message Parsers}
%
%    \begin{macrocode}
\def\annMessagePunct{}
%    \end{macrocode}
% observe the punctuation:
%    \begin{macrocode}
\newcommand{\testMessagePunct}{%
  \IfEndWith{\lymessage}{.}
    {\def\annMessagePunct{.}}
    {\IfEndWith{\lymessage}{?}
      {\def\annMessagePunct{?}}
      {\IfEndWith{\lymessage}{!}
        {\def\annMessagePunct{!}}
        {\def\annMessagePunct{\unskip}}}}}
%    \end{macrocode}
% parse messages:
%    \begin{macrocode}
\newcommand{\annMessageParse}{%
  \IfEndWith{\lymessage}{.}
    {\StrGobbleRight{\lymessage}{1}[\lymessage]
      \annMessageOO{\annMessageO{%
        \annMessage{\prependMessage\lymessage}}}\nobreak}
    {\IfEndWith{\lymessage}{?}
      {\StrGobbleRight{\lymessage}{1}[\lymessage]
        \annMessageOO{\annMessageO{%
          \annMessage{\prependMessage\lymessage}}}\nobreak}
      {\IfEndWith{\lymessage}{!}
        {\StrGobbleRight{\lymessage}{1}[\lymessage]
          \annMessageOO{\annMessageO{%
            \annMessage{\prependMessage\lymessage}}}\nobreak}
        {\ifthenelse{\equal{\lymessage}
          {no value given}}{\unskip\noMessError}
          {\ifthenelse{\equal{\lymessage}{}}{\unskip\noMessError}
            {\annMessageOO{\annMessageO{%
              \annMessage{\prependMessage\lymessage}}}\nobreak}}}}}}
%    \end{macrocode}
% set the append/punct vs. punct/append orientations:
%    \begin{macrocode}
\newcommand{\annMessTail}{%
  \annMessageOO{\annMessageO{%
    \annMessage{\annMessagePunct\appendMessage}}}}
%    \end{macrocode}
% macro to set order: punct, append
%    \begin{macrocode}
\newcommand{\annMessPunctAppend}{%
  \renewcommand{\annMessTail}{%
    \annMessageOO{\annMessageO{%
      \annMessage{\annMessagePunct\unskip\appendMessage}}}}}
%    \end{macrocode}
% macro to set order: append, punct
%    \begin{macrocode}
\newcommand{\annMessAppendPunct}{%
  \renewcommand{\annMessTail}{%
    \annMessageOO{\annMessageO{%
      \annMessage{\appendMessage\unskip\annMessagePunct}}}}}
%    \end{macrocode}
% get type from key
%    \begin{macrocode}
\def\annTypeName{%
  \StrSubstitute{\lyanntype}{-}{ }[\annTypeCleaned]
  \prependType\titlecap{\annTypeCleaned}\appendType}
%    \end{macrocode}
% store original error handling for unknown keys
%    \begin{macrocode}
\let\KV@errx@ORI\KV@errx
%    \end{macrocode}
%
% \subsection{Default Annotation Parser}
%
%    \begin{macrocode}
\newcommand\annotationParse[1]{
  \resetkeys
    \let\KV@errx\@gobble
    \setkeys{scholarLYann}{#1}
    \let\KV@errx\KV@errx@ORI
\saveexpandmode\expandarg
  \testMessagePunct
  \processGrobLocation
  \def\currentMeasure{\lymeasureno}
  \def\currentBeat{\lyourbeat}
  \testLocation
  \itemornot
    \annTypePrePrefix\annTypePfx
      \annName\annPostType\typeSkipornot
    {\annArgsPrePrefix\annLocationPfx
      {\annMeasure{\thisMeasure}}}
    {\annBeat{\thisBeat}\skipornot}
    \ifthenelse{\equal{\lycontextid}{}}
      {\unskip{}}
      {\annArgsPrePrefix\annContextPfx
        \annVoiceO{\annVoice{\lycontextid}}\skipornot}
    \ifthenelse{\equal{\lygrobtype}{}}
      {\unskip{}}
      {\annArgsPrePrefix\annGrobTypePfx
        \annAffectedO{\annAffected{\lygrobtype}}\skipornot}
    \annArgsPrePrefix\annMessagePfx{{{%
      \prependMessageFirst\annMessageParse
        \unskip\annMessTail}\unskip}
      \ifthenelse{\equal{\lyannfootnote}{no value given}}
        {}
        {\unskip{\footnote{\lyannfootnote}}}\appendMessageLast}
  \StrDel{\lymeasureno}{ }[\previousMeasure]
  \StrDel{\lyourbeat}{ }[\previousBeat]}
%    \end{macrocode}
% annotation macros:
%    \begin{macrocode}
\newcommand{\criticalRemark}[1][]{\annotationParse{#1}}
\newcommand{\musicalIssue}[1][]{\annotationParse{#1}}
\newcommand{\lilypondIssue}[1][]{\annotationParse{#1}}
\newcommand{\annotateQuestion}[1][]{\annotationParse{#1}}
\newcommand{\annotateTodo}[1][]{\annotationParse{#1}}
\newcommand{\annotation}[1][]{\annotationParse{#1}}
%    \end{macrocode}
%
% \subsection{Custom Annotations}
%
% This is a TODO. The idea with custom annotations is to allow end users to
% define their own annotation formats entirely while still having access to
%  the infrastructure provided by the bulk of this package (i.e. input file
% parsing, style blocks, and other useful formatting features).
%
%
%\iffalse
%</package>
%\fi
% \Finale
\endinput
